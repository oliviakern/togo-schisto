 
\documentclass[12pt]{article}
 
\usepackage[margin=1in]{geometry} 
\usepackage{amsmath,amsthm,amssymb}
%\usepackage[mathscr]
%\usepackage{mathrsfs}
\begin{document}
 

% --------------------------------------------------------------
 
 
\title{Modeling Schistosomiasis Transmission in Togo's Ogou District}%replace X with the appropriate number
\author{Olivia Kern\\ %replace with your name
Fulbright Student | Togo, 2017} %if necessary, replace with your course title
 
\maketitle
 
Here I apply a set of ordinary differential equations adapted from Soklow et al. 2017 to which I have added ($\iota$,y,z) to reflect local dynamics observed during fieldwork contributing to Schistosomiasis transmission in Ogou. I also developed an equation to describe the likelihood of mass drug administration (MDA) success as a function of the variables $f$, $a$, $p$, $unknown$ affecting treatment compliance.

presence and awake, history of adverse reactions, having eaten something on any given distribution day 
treatable population is defined as modified by incorporating (iota and ?) to fit the particular setting in Togo's Ogou district. I incorporated ... term into Eq (\ref{eq:PopRateofChange})  to reflect a consistent number of Ghanian immigrants belonging to the () culture who may have joined the community without having received preventative chemotherapy, increasing the susceptible human population
phenomenon observed during fieldwork. 
Adapted a nonlinear model  accounting for snail dispersal


The first equation (\ref{eq:PopRateofChange}) describes the rate of human population change in endemic villages. the per capita natality rate $\mu_H$ represents population growth, to which I incorporated a variable $\iota$ representing net migration given the importance of an influx of untreated persons to the transmission dynamics
 offset by parasite-incduced death given $\alpha$ parasitic pathogenicity.
including the mating pair dynamics (?) and the impact of poor sanitation (variable)
This $N_v$ is the symbol for the population...

\begin{equation}\label{eq:PopRateofChange}
\dot{N}_v = \mu_H\iota(H_v - N_v) - \alpha P_v
\end{equation}
 
 \begin{equation}\label{eq:ParasiteRateofChange}
 \dot{P}_v = F_vN_v - (\mu_H + \mu_P + \alpha)P_v - \alpha \frac{k +1}{k} \frac{P_v^2}{N_v}
 \end{equation}

\begin{equation}\label{eq:SusceptibleRateofChange}
\dot{S}_w = vS_w [1- \gamma_w (S_w + E_w + I_w)] - \mu_S S_w - \rho M_wS_w + D_w^s
 %$\mathscr{D}$
\end{equation}
 
\begin{equation}\label{eq:ExposedRateofChange}
\dot{E}_w = \rho M_wS_w - (\mu_s +\eta) I_w + D_w^E
\end{equation}
 
\begin{equation}\label{eq:InfectedRateofChange}
\dot{I}_w = \delta E_w - (\mu_s +\eta) I_w + D_w^I
\end{equation}
  
\begin{equation}\label{eq:CercariaeRateofChange}
\dot{C}_w = \zeta I_w - \mu_cC_w+  L_w^I
 %$\mathscr{L}$
\end{equation}
 
\begin{equation}\label{eq:MiracidiaRateofChange}
\dot{M}_w = \varrho M_w +  L_w^M
 %$\mathscr{L}$
\end{equation}
 
\begin{equation}\label{eq:ForceofInfection}
 F_v = \beta \sum_{j=1}^{n_w} \Omega_vw C_w
\end{equation}

\begin{equation}\label{eq:contact matrix}
 F_v = \beta \sum_{j=1}^{n_w} \Omega_vw C_w
\end{equation}

$n_v$ villages

$n_w$ water points

$N_v$ human population size in each village

$P_v$ total number of parasites in human hosts in each village

$S_w$ susceptible snails in freshwater source

$E_w$ exposed snails in freshwater source

$I_w$ infected snails in freshwater source

$C_w$ concentration of cercariae in freshwater source

$M_w$  concentration of miracidia in freshwater source

$P_v$ total number of adult parasites

$\mu_H$ natural human host mortality

$\mu_P$ per capita parasite mortality

$v$ intrinsic natality rate

$\gamma_w$ site-specific effect of density dependence

$\mu_s$ mortality rate of susceptible snails

%ρ: exposure rate of susceotible snails

%1/δ: delay between infection (S-> E) and onset of infectiousness (E -> I)

%η: disease-induced death rate infected snails are subject to

%μ_C: uniform mortality rate of cercariae

%ζ: rate at which cercariae are shed from infected snails

%μ_M: uniform mortality rate of miracidia

χ: human contamination rate
 % --------------------------------------------------------------
\begin{equation}
\rho \cap \alpha
\end{equation}


\section{Equation Example}

A typical Linear-Time invariant (LTI) system can be described by the equation

\begin{equation}\label{eq:ltiSys}
\dot{\mathbf{x}}(t) = \mathbf{A} \mathbf{x}(t) + \mathbf{B} \mathbf{u}(t),
\end{equation}

in which ${\mathbf{x}}(t)$ represents the system's state vector, $\mathbf{A}$ denotes the system matrix that models the interaction between variables, $\mathbf{B}$ the way in which the inputs affect the different states, and $\mathbf{u}(t)$ the chosen control action at time $t$.

This way of expressing an LTI system yields a deterministic equation. Since I am interested in working with stochastic processes, I chose to add an additional term to this equation. The modified equation  looks like

\begin{equation}\label{eq:ltiSysStochastic}
\dot{\mathbf{x}}(t) = \mathbf{A} \mathbf{x}(t) + \mathbf{B} \mathbf{u}(t) + \underbrace{ \mathbf{B}_w \mathbf{w}(t)}_{\text{random noise variable}},
\end{equation}

in which the additional term is a factor of two variables:
\begin{align*}
\mathbf{w}(t) \in \mathbb{R}^{n} &: \text{white uncorrelated random noise variable} \\
\mathbf{B}_w \in \mathbb{R}^{m\times n} &: \text{matrix that denotes which states are corrupted by noise}
\end{align*}


The reason behind the addition of the variables $\mathbf{w}(t)$ is that in real life scenarios, our sensors are not able to tell us with $100\%$ accuracy what the actual state of our variables are. Ancillary, if we are controling a system, it is unavoidable that the control input commanded from our computers --- e.g., the force or torque to be produced at a certain axis --- will not be exactly the same as the one produced by our actuators, and there is bound to be some uncertainty there as well. By including this random variable in the equation, and giving it the appropriate noise characteristics and behaviour that accuratelly describes the statistical behavior of our system, we can obtain a stochastic model.

 
\end{document}